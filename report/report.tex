\documentclass[11pt,english]{article}

\usepackage[T1]{fontenc} \usepackage[latin9]{inputenc} \usepackage{verbatim} \usepackage{float} \usepackage{amsmath} \usepackage{amssymb} \usepackage{graphicx} \makeatletter %%%%%%%%%%%%%%%%%%%%%%%%%%%%%% LyX specific LaTeX commands.  %% Because html converters don't know

%%%%%%%%%%%%%%%%%%%%%%%%%%%%%% User specified LaTeX commands.
\usepackage{amssymb, amsthm, geometry, graphicx, caption, enumerate, mathrsfs}
\newgeometry{margin=1in}
\usepackage{fancyhdr}

\makeatletter
\renewcommand*\env@matrix[1][*\c@MaxMatrixCols c]{%
\hskip -\arraycolsep
\let\@ifnextchar\new@ifnextchar
\array{#1}}
\makeatother
\pagestyle{fancy}

\setlength\parindent{0pt}  % no indents

% Reset section number in each part.
\usepackage{chngcntr}
\counterwithin*{section}{part}

\renewcommand{\theenumi}{\alph{enumi}} \renewcommand{\theenumii}{\arabic{enumii}} \renewcommand{\labelenumii}{\theenumii}

\makeatother

\usepackage{babel}


\title{Surveillance Exploration}
\author{}
\date{\today}

\lhead{}
\chead{}
\rhead{}
\lfoot{}
\cfoot{\thepage}
\rfoot{}
\renewcommand{\headrulewidth}{0pt}
\renewcommand{\footrulewidth}{0pt}

\newcommand{\beq}{\begin{equation*} \begin{aligned}}
\newcommand{\eeq}{\end{aligned} \end{equation*}}


\begin{document}
\maketitle

\newcommand{\proj}{\mathcal{P}}
\newcommand{\backproj}{\mathcal{Q}}



We consider the problem of patrolling a known environment while maintaining surveillance
of a target.




\section{Surveillance}

\section{Patrol}


Let $\mathcal{X}$ be the set of all possible environment configurations.
Each $\Omega \in \mathcal{X}$ is an open set representing the free space
and $\Omega^C$ is a closed set consisting of a finite number of connected components.
Let $\mathcal{O} = \{x_i\}_{i=0}^k$ be the sequence of vantage points. For
each vantage point, the operator $\proj_{x_i}\Omega$ is a projection of
$\Omega$ along $x_i$. Then $\proj_{x_i}\Omega$ is a set of range measurements defined
on the unit sphere.  The back projection $\backproj$ maps the range
measurements to the visibility set
$\mathcal{V}_{x_i} \Omega := \backproj (\proj_{x_i}) \Omega$;
that is, points in this set are visible from $x_i$.
As more range measurements are acquired, the
environment can be approximated by the \emph{cumulatively visible set} $\Omega_k$:
$$\Omega_k = \bigcup_{i=0}^k \mathcal{V}_{x_i}\Omega \ . $$

By construction, $\Omega_k$ admits partial ordering: $\Omega_{i-1} \subset \Omega_{i}$.
For suitable choices of $x_i$, it is possible that $\Omega_n \to \Omega$,
(say, in the Hausdorff distance).  We aim at determining a \emph{minimal number
of  vantage points}  from which every point in $\Omega$ can be seen.


\subsection{A Greedy Approach}
We consider a greedy approach which sequentially determines a new vantage point, $x_{k+1}$, based on the information gathered from all previous vantage points, $x_0,x_1,\cdots, x_{k}$.
The strategy is greedy because $x_{k+1}$ would be a location that \emph{maximizes the information gain}.

If the environment $\Omega$ is known, we define the \emph{gain} function
$$g(x;\Omega_k, \Omega) := |   \mathcal{V}_x \Omega \cup \Omega_k| - |\Omega_k |, $$

i.e. the volume of the region that is visible from $x$ but not from $x_0,x_1,\cdots,x_{k}$. 

Define $\Gamma$ as the feasible set (satisfying the surveillance requirements).
Then, for patrolling, we consider:
\begin{equation} x_{k+1} = \arg \max_{x\in \Gamma} g(x;\Omega_k, \Omega).\end{equation}
In other words, the next vantage point should be the point in the feasible set that maximizes the newly surveyed area. 

\end{document}
